\section{truerand.c File Reference}
\label{truerand_8c}\index{truerand.c@{truerand.c}}


{\tt \#include $<$signal.h$>$}\par
{\tt \#include $<$setjmp.h$>$}\par
{\tt \#include $<$sys/time.h$>$}\par
{\tt \#include $<$math.h$>$}\par
{\tt \#include $<$stdio.h$>$}\par
{\tt \#include $<$sys/types.h$>$}\par
{\tt \#include $<$sys/stat.h$>$}\par
{\tt \#include $<$fcntl.h$>$}\par
{\tt \#include $<$unistd.h$>$}\par


Include dependency graph for truerand.c:\subsection*{Functions}
\begin{CompactItemize}
\item 
unsigned long {\bf raw\_\-truerand} ()
\item 
int {\bf raw\_\-n\_\-truerand} ({\bf n}) int {\bf n}
\end{CompactItemize}


\subsection{Function Documentation}
\index{truerand.c@{truerand.c}!raw_n_truerand@{raw\_\-n\_\-truerand}}
\index{raw_n_truerand@{raw\_\-n\_\-truerand}!truerand.c@{truerand.c}}
\subsubsection{\setlength{\rightskip}{0pt plus 5cm}int raw\_\-n\_\-truerand ({\bf n})}\label{truerand_8c_a8}


\index{truerand.c@{truerand.c}!raw_truerand@{raw\_\-truerand}}
\index{raw_truerand@{raw\_\-truerand}!truerand.c@{truerand.c}}
\subsubsection{\setlength{\rightskip}{0pt plus 5cm}unsigned long raw\_\-truerand ()}\label{truerand_8c_a7}




Definition at line 135 of file truerand.c.

References buf, and qshs().



\footnotesize\begin{verbatim}136 {
137         void (*oldalrm)();
138         struct itimerval it;
139         unsigned long counts[12];
140         unsigned char *qshs();
141         unsigned char *r;
142         unsigned long buf;
143         int i;
144 
145         getitimer(ITIMER_REAL, &it);
146         oldalrm = signal(SIGALRM, SIG_IGN);
147         for (i=0; i<12; i++) {
148                 counts[i]=0;
149                 while ((counts[i] += roulette()) < 512)
150                         ;
151         }
152         signal(SIGALRM, oldalrm);
153         setitimer(ITIMER_REAL, &it, NULL);
154 
155         r = qshs(counts,sizeof(counts));
156         buf = *((unsigned long *) r);
157 
158         return buf;
159 }
\end{verbatim}\normalsize 


Here is the call graph for this function: